\documentclass{matematicasud}



%% COMANDOS PERSONALIZADOS
	%% CONJUNTOS ESPECIALES
\newcommand{\R}{\mathbb{R}}

    %% OPERADOERES MATEMÁTICOS
\DeclareMathOperator{\dom}{dom}

%% LOS AMBIENTES SE LLAMAN CON \begin{teorema}..\end{teorema} ETC. LOS NOMBRES DE LOS AMBIENTES SON: teorema, lema, proposicion, corolario, definicion, nota y ejercicio.

%% ESCRIBA AQUÍ EL TÍTULO DEL TRABAJO
\titulodeltrabajo{Transformaciones de Möbius: Una mirada al análisis complejo}

%% NOMBRE DE LOS AUTORES Y EL DIRECTOR PRIMER ARGUMENTO AUTORES Y SEGUNDO EL DIRECTOR
\autoresydirector{Jeisson Jarvey Sánchez Castellanos}{Álvaro Arturo Sanjuán Cuéllar}

%% El MES Y EL AÑO DE LA PRESENTACIÓN DEL TRABAJO PRIMER ARGUMENTO EL MES Y EL SEGUNDO EL AÑO.
\mesyano{septiembre}{2023}

\begin{document}

\maketitle


\abstract{
En este trabajo vamos a estudiar los aspectos básicos de las transformaciones de Möbius, para esto expondremos la notación de circunferencias fundamentales, su asociación con las transformaciones y algunas propiedades elementales.\cite{schwerdtfeger1979geometry}

El objetivo de este trabajo es mostrar de manera resumida la conservación de las formas que lleva las transformaciones de Möbius, mostrando a su vez que estas transformaciones son bilineales y proyectivas observando que las mismas se transforman en el plano complejo y en la esfera de Riemann. \cite{ayres1971geometria}, p 172

}
\textbf{Palabras clave: } Transformación de Möbius, Matriz asociada, circunferencia fundamental, inversión.

\textbf{Clasificación AMS:} 30C15

\textbf{Agradecimientos: } Agradezco a Dios por darme la paciencia y fortaleza para finalizar este proyecto, a mi madre que me ha dado la vida y por ser mi apoyo financiero, a mi tutor por ser mi guía tanto académico como en la vida, a mi pareja por ser mi apoyo emocional y a mis compañeros por ser parte de mi crecimiento profesional.

\newpage






\section{Introducción}\label{cap:1}

Cada vez que en matemáticas se aborda una teoría particular se hace necesario establecer un lenguaje, un conjunto de notaciones que nos indican de manera práctica que son los elementos y como se manejan. Para este trabajo realizamos un análisis en los números complejos. Para facilidad del lector, hacemos uso del plano cartesiano como la composición del eje real y el eje imaginario, de modo que el punto $(x,y)$ esta relacionado con el número complejo $z$ por la siguiente ecuación $z = x + iy$ donde $i$ es el valor complejo $i =\sqrt{-1}$. El valor de $x = \Re(z)$ es la parte real de $z$ e $y = \Im(z)$ es la parte imaginaria de $z$. El conjugado de $z$ es una proyección simétrica con respecto al eje real, de modo que $\bar{z} = x-iy$ es el conjugado de $z$. El módulo de $z$ denotado como $|z| = \sqrt{x^2+y^2}$ es la distancia del punto $z$ al origen.

De manera similar podemos definir el número complejo $z \neq 0$ en coordenadas polares como sigue
$$z = r(\cos{\theta} + i\sin{\theta})$$
donde $r$ y $\theta$ son dos valores reales no negativos y $\cos{\theta} + i \sin{\theta}$ es llamado factor angular de $z$. Este es un número complejo de módulo uno denotado por $e^{i\theta}$. Es necesario tener en cuenta los conceptos básicos del álgebra lineal, debido a que en los temas a tratar hacemos uso de asociación de funciones o circunferencias con matrices. Sin embargo, nos restringimos a matrices $2 \times 2$ de la forma

\begin{equation*}
    \mathfrak{H} = 
    \begin{pmatrix}
        a & b \\
        c & d \\
    \end{pmatrix}
    \hspace{1cm}
    \mathfrak{C} =
    \begin{pmatrix}
        A & B\\
        C & D\\
    \end{pmatrix}.
\end{equation*}
La primera matriz está asociada a una transformación de Möbius, la segunda esta asociada a una circunferencia fundamental.

\newpage

\section{Circunferencias y matrices asociadas a las transformaciones de Möbius}\label{cap:2}

Decimos que una circunferencia consta de todos los puntos que equidistan de un punto el cual es denominado como centro de la circunferencia, esto ocurre en lo que llamaremos un $z$-plano. En la geometría cartesiana, tomábamos los puntos $(x,y)$ que equidistan de un centro $(a,b)$ a una distancia $\rho$ y obteníamos la formula $(x-a)^2+(y-b)^2 = \rho^2$. Cambiamos las variables a la notación de números complejos de la siguiente manera $(x,y)=x+iy=z$ y $(a,b) = a + ib =\gamma$.

Tenemos entonces que el radio $\rho$ es la distancia entre cualquier $z$ y el punto $\gamma$, es decir, $|z-\gamma|=\rho$, también que $|z-\gamma|^2=\rho^2$. Ahora bien, recordando las propiedades del módulo de números complejos, obtenemos $|z-\gamma|^2=(z-\gamma)(\bar{z}-\bar{\gamma})=z\bar{z}-\bar{\gamma}z-\gamma\bar{z}+\gamma\bar{\gamma}$, finalmente llegamos a la ecuación de la circunferencia con centro $\gamma$ y radio $\rho$
\begin{equation}
    z\bar{z}-\bar{\gamma}z-\gamma\bar{z}+\gamma\bar{\gamma}-\rho^2=0.
    \label{eq:(1)}
\end{equation}
Para entender la asociación de \eqref{eq:(1)} con una matriz. es necesario ver su forma general
\begin{equation}
    \mathfrak{C}(z,\bar{z})=Az\bar{z}-Bz-C\bar{z}+D=0,
    \label{eq:(2)}
\end{equation}

donde $A$ y $D$ son reales y $B$ y $C$ son complejos conjugados. Es decir, la matriz
\begin{equation}
    \mathfrak{C}=
    \begin{pmatrix}
        A & B \\
        C & D 
    \end{pmatrix}
    \label{eq:(3)}
\end{equation}
es una matriz hermitiana.
Cabe resaltar que la ecuación \eqref{eq:(1)} es un caso particular de la ecuación \eqref{eq:(2)} y por ende representan la misma circunferencia sí $A\neq 0$, no es difícil de ver pues
\begin{equation}
    B=-A\bar{\gamma};\quad C=-A\gamma = \bar{B}; \quad D = A(\gamma\bar{\gamma}-\rho^2). 
    \label{eq:(4)}
\end{equation}
Cada matriz $\mathfrak{C}$ hermitiana está asociada a una ecuación de la forma \eqref{eq:(2)} y representa una circunferencia excepto cuando $A=B=C=0$.

Fijémonos en la ecuación \eqref{eq:(4)}. Al detallar que es la misma circunferencia de la ecuación \eqref{eq:(1)}, podemos darnos cuenta que si tenemos dos matrices hermitianas $\mathfrak{C}_1,\mathfrak{C}_2$, estas representan la misma circunferencia sí y solo sí $\mathfrak{C}_1=\lambda\mathfrak{C}_2$, donde $\lambda$ es un real diferente de cero. Para determinar los diferentes tipos de circunferencias incluidos en una misma matriz hermitiana usaremos su determinante
\begin{equation}
    |\mathfrak{C}|=\Delta = AD-BC=AD-|B|^2.
    \label{eq:(5)}
\end{equation}
Evidentemente $\Delta$ es un número real, que se conoce como el discriminante de la circunferencia $\mathfrak{C}$. Si observamos la ecuación \eqref{eq:(1)}, vemos que el discriminante de circunferencia es $\delta = -\rho^2$, para la circunferencia dada por la ecuación \eqref{eq:(2)}, y que le corresponde la matriz de la ecuación \eqref{eq:(3)}, tiene por discriminante 
\begin{equation}
    \Delta = -A^2\rho^2
    \label{eq:(6)}
\end{equation}
Ahora se hace evidente que son la misma circunferencia, pues cada discriminante esta ligado al otro por un producto escalar.

Nótese que siempre que $A \neq 0$ y $\Delta < 0$ tenemos que $\mathfrak{C}$ es una circunferencia real ordinaria. En el caso de que $A=0$ la ecuación general \eqref{eq:(2)} se convierte en una ecuación lineal, de modo que obtenemos la ecuación de la recta con coeficientes reales. Para el caso de que $A \neq 0$ pero $\Delta = 0$ los puntos aparecen como un punto circular, pues $\rho = 0$. Finalmente, podemos ver que pasa cuando $\Delta > 0$. De acuerdo con la ecuación \eqref{eq:(6)}, observamos que $\rho^2 < 0$. De manera que tenemos una circunferencia imaginaria con radio imaginario $\rho$ y un centro real $\gamma$. Una circunferencia imaginaria no tiene puntos reales. Es decir que no existen $x$ ni $y$ reales tales que $z=x+iy$ y $z$ sea un punto de la circunferencia. Sin embargo, su radio puede ser hallado con las ecuaciones \eqref{eq:(4)} y \eqref{eq:(5)}.

Veamos el ejemplo de la circunferencia imaginaria unidad dado por la siguiente ecuación y matriz
\begin{equation}
    z\bar{z}+1=0 \quad \mathfrak{C}=
    \begin{pmatrix}
        1 & 0 \\
        0 & 1
    \end{pmatrix}.
    \label{eq:(7)}
\end{equation}
Esta ecuación no puede ser resuelta por un número complejo ordinario $z$. Sin embargo, podemos  escribir la circunferencia en la forma $x^2+y^2=-1$ y esta ecuación puede resolverse por puntos en el espacio, cuyas coordenadas $x,y$ no son ambos números reales. Una solución trivial sería $x=0$ y $y=i$.

La circunferencia $\mathfrak{C}$ se dice que tiene una orientación positiva si $A>0$, es decir $-\mathfrak{C}$ es la misma circunferencia con orientación negativa. Esta orientación puede ser presentada por una flecha en la circunferencia que va en sentido antihorario si es positiva. Esta noción de orientación puede llevarse al caso de circunferencias imaginarias, pues esta sigue siendo determinada por el signo de $A$.

Sea $\mathfrak{C}_1$ y $\mathfrak{C}_2$ representaciones de dos circunferencias diferentes. Es decir, que son matrices no proporcionales la una de la otra. Sean $\lambda_1$ y $\lambda_2$ dos números reales no ambos cero, definimos entonces a $\mathfrak{C}=\lambda_1\mathfrak{C}_1+\lambda_2\mathfrak{C}_2$ como un haz de circunferencias.

En geometría proyectiva \cite{halsted1896}, un haz es una familia de objetos geométricos con una propiedad común. Por ejemplo el conjunto de las circunferencias de Apolonio (Figura \ref{fig:Apolonio}), podemos apreciar dos haces ortogonales de circunferencias.

\begin{figure}[!ht]
    \begin{center}
        \begin{tikzpicture}
            \clip(-4,-4) rectangle (8,4);
            \draw [line width=2pt] (2,1.5) circle (2.5cm);
            \draw [line width=2pt] (2,3.75) circle (4.25cm);
            \draw [line width=2pt] (2,2.2916666666666665) circle (3.0416666666666665cm);
            \draw [line width=2pt] (2,0.5833333333333334) circle (2.0833333333333335cm);
            \draw [line width=2pt] (2,0.975) circle (2.225cm);
            \draw [line width=2pt] (2,-3.75) circle (4.25cm);
            \draw [line width=2pt] (2,-2.2916666666666665) circle (3.0416666666666665cm);
            \draw [line width=2pt] (2,-1.5) circle (2.5cm);
            \draw [line width=2pt] (2,-0.975) circle (2.225cm);
            \draw [line width=2pt] (2,-0.5833333333333334) circle (2.0833333333333335cm);
            \draw [line width=2pt] (4,0) circle (0.9381381505634503cm);
            \draw [line width=2pt] (0,0) circle (0.9381381505634503cm);
        \end{tikzpicture}
    \end{center}
    \caption{Haz de circunferencias de Apolonio}
    \label{fig:Apolonio}
    \end{figure}

Haciendo uso de la definición del haz de circunferencias, podemos ver que el determinante está dado por
\begin{align}
    |\mathfrak{C}|&=
    \begin{vmatrix}
        \lambda_1A_1+\lambda_2A_2 & \lambda_1B_1+\lambda_2B_2\\
        \lambda_1C_1+\lambda_2C_2 &
        \lambda_1D_1+\lambda_2D_2
    \end{vmatrix}\\
    &=\Delta_1\lambda^2_1+2\Delta_{12}\lambda_1\lambda_2+\Delta_2\lambda^2_2.
    \label{eq:(9)}
\end{align}
La ecuación \eqref{eq:(9)} es una forma cuadrática en las variables $\lambda_1$ y $\lambda_2$ con coeficientes reales donde 
\begin{equation}
    \Delta_1=|\mathfrak{C}_1|,\quad\Delta_2=|\mathfrak{C}_2|,\quad 2\Delta_{12}=A_1D_2+A_2D_1-B_1C_2-B_2C_1.
    \label{eq:(10)}
\end{equation}
Ahora bien, sea $A_1$ y $A_2 \neq 0$, sí $\mathfrak{C}_1,\mathfrak{C}_2$ son las circunferencias $(\gamma_1,\rho_1),(\gamma_2,\rho_2)$, entonces por la ecuación \eqref{eq:(3)} y la ecuación \eqref{eq:(6)} tenemos

\begin{equation}
    \Delta_1=-A_1^2\rho_1^2 \quad \Delta_2=-A_2^2\rho^2_2 \quad 2\Delta_{12}=A_1A_2(\delta^2-\rho^2_1-\rho^2_2),
    \label{eq:11}
\end{equation}
donde $\delta =|\gamma_1-\gamma_2|$ es la distancia entre los centros de cada circunferencia.

Supongamos que $\mathfrak{C}_1$ y $\mathfrak{C}_2$ son ambas circunferencias reales y que ellas tienen al menos un punto real en común (Figura \ref{fig:representacion}). Entonces el ángulo $\omega$ entre las dos direcciones de las $\mathfrak{C}_1$ y $\mathfrak{C}_2$ está definido como el ángulo formado por las tangentes de las dos circunferencias en el punto común en la dirección definida por la orientación.

Por el teorema del coseno de geometría elemental tenemos que 
\begin{equation}
    \delta^2=\rho^2_1+\rho^2_2\mp2\rho_1\rho_2\cos{\omega}.
    \label{eq:(12)}
\end{equation}
Por lo tanto,
\begin{equation}
    2\Delta_{12}=\mp2A_1A_2\rho_1\rho_2\cos{\omega}=-2\sqrt{-\Delta_1}\sqrt{-\Delta_2}\cos{\omega}.
    \label{eq:(13)}
\end{equation}
El símbolo es determinado únicamente por las orientaciones de las dos circunferencias $\mathfrak{C}_1$, $\mathfrak{C}_2$ y 
\begin{equation}
    \cos{\omega}=-\frac{\Delta_{12}}{\sqrt{-\Delta_1}\sqrt{-\Delta_2}}=\frac{\Delta_{12}}{\sqrt{\Delta_1}\sqrt{\Delta_2}}.
    \label{eq:(14)}
\end{equation}
Para las circunferencias reales es necesario y suficiente que $\omega$ sea un ángulo real. Es decir
\begin{equation*}
    -1 \leq \cos{\omega} \leq +1,
\end{equation*}

\begin{figure}[!ht]
    \begin{center}
        \begin{tikzpicture}
            \clip(-7.6342439689235615,-4.368371902504899) rectangle (7.553338099925109,5.798264143582682);
            \draw [shift={(-1.870606344795017,3.5356515527993744)},line width=2pt] (0,0) -- (-152.11803641712876:0.4677900021205956) arc (-152.11803641712876:-89.86680937764368:0.4677900021205956) -- cycle;
            \draw [shift={(-1.870606344795017,3.5356515527993744)},line width=2pt] (0,0) -- (-62.118036417128764:0.4677900021205956) arc (-62.118036417128764:0.133190622356307:0.4677900021205956) -- cycle;
            \draw [shift={(1.8706063447950163,3.5356515527993744)},line width=2pt] (0,0) -- (89.86680937764365:0.4677900021205956) arc (89.86680937764365:152.11803641712876:0.4677900021205956) -- cycle;
            \draw [line width=2pt] (0,0) circle (4cm);
            \draw [line width=2pt] (0,0)-- (-1.870606344795017,3.5356515527993744);
            \draw [line width=2pt] (-1.870606344795017,3.5356515527993744)-- (0,3.54);
            \draw [line width=2pt] (0,3.54)-- (0,0);
            \draw [line width=2pt] (0,3.54) circle (1.8706113990298547cm);
            \draw [-latex',line width=2pt] (-1.870606344795017,3.5356515527993744) -- (-3.3442043453687043,2.7560154013330633);
            \draw [-latex',line width=2pt] (-1.870606344795017,3.5356515527993744) -- (-1.8670372584060893,2.000309087726063);
            \draw [-latex',line width=2pt] (0,5.410611399029855) -- (-0.2621903792280043,5.392145569662576);
            \draw [-latex',line width=2pt] (0,4) -- (-0.20747860695070913,3.994615454290662);
                \begin{scriptsize}
                \draw [fill=black] (0,0) circle (2pt);
                \draw[color=black] (0.27140706691450406,0.30173161866569925) node {$\gamma_1$};
                \draw [fill=black] (0,3.54) circle (2.5pt);
                \draw[color=black] (0.19344206656107146,3.7569356347822152) node {$\gamma_2$};
                \draw[color=black] (-1.319078940295521,1.767473625310228) node {$\rho_1$};
                \draw[color=black] (0.27140706691450406,2.172891627148076) node {$\delta$};
                \draw[color=black] (-2,3.3) node {$\omega$};
                \draw[color=black] (-1.2411139399420876,3.2) node {$\omega$};
                \draw[color=black] (1.7,3.8) node {$\omega$};
                \end{scriptsize}
        \end{tikzpicture}
    \end{center}
    \caption{Representación de $\mathfrak{C}_1$ y $\mathfrak{C}_2$}
    \label{fig:representacion}
    \end{figure}
lo que equivale a decir que $\Delta_1\Delta_2-\Delta_{12}^2 \geq 0$ con $\Delta_1 < 0$ es la condición para que la forma cuadrática \eqref{eq:(9)} tenga un valor no positivo para todo real $\lambda_1$ y $\lambda_2$.

\textbf{Inversión} Sea $\mathfrak{C}$ una circunferencia y $z$ un punto que no está en el interior de la circunferencia $\mathfrak{C}$ y es diferente a su centro. Entonces hay un único punto $z^*$ diferente de $z$ que es común a todas las circunferencias que pasan por $z$ que son ortogonales a $\mathfrak{C}$.

Lo anterior es un teorema que define lo que es un punto inverso respecto a una circunferencia fundamental. Ahora bien, cada punto en el plano, excepto el centro $z=\gamma=-C/A$ de una circunferencia fundamental, es llevado por la inversión en un punto definido en la imagen de $z^*$ por
\begin{equation}
    Az^*\bar{z}+Bz^*+C\bar{z}+D=\mathfrak{C}(z^*,\bar{z}).
    \label{eq:(15)}
\end{equation}
En este caso, $z^*$ está definido de manera única como función de $z$
\begin{align}
    z^*&=-\frac{C\bar{z}+D}{A\bar{z}+B}\notag\\
    &=\gamma+\frac{\rho^2}{\bar{z}-\bar{\gamma}}.
    \label{eq:(16)}
\end{align}
Sea $z^*=f(z)$ una relación. Con el fin de hacer la inversión una correspondencia uno a uno (biyectiva) de todo el plano sobre sí mismo, es necesario completar el plano de los números complejos ordinarios añadiéndole un elemento más que se llama el punto infinito y es representado por el símbolo $\infty$.
De este modo, completamos la definición de inversión poniendo
\begin{equation}
    f(\gamma)=\infty
    \label{eq:(17)}
\end{equation}
pues la inmersión es involutiva, esto es, $f(z^*)=z$. De modo que es natural poner
\begin{equation}
    f(\infty)=\gamma,
    \label{eq:(18)}
\end{equation}
con lo cual podemos decir que $z^*=f(z)$ es una correspondencia del plano complejo en sí mismo. De \eqref{eq:(15)} o \eqref{eq:(16)} se deriva
\begin{equation*}
    \lim_{z \rightarrow \gamma}f(z)=\infty, \quad \lim_{z \rightarrow \infty}f(z)=\gamma
\end{equation*}
que con respecto a \eqref{eq:(17)} y \eqref{eq:(18)} expresa la continuidad de la inversión en el centro $\gamma$ y en $\infty$ respectivamente. Procedemos entonces a hacer énfasis en lo que podría considerarse como el corazón de este trabajo.

Se dice que una transformación $Z=\phi(z)$ del $z$-plano en el $Z$-plano es \textit{isogonal} en el punto $z_0$, si envía dos curvas cualesquiera que se encuentran en $z_0$ formando el ángulo $\omega$, en dos curvas formando el ángulo $\Omega=\pm\omega$ en $Z_0=\phi(z_0)$. Se dice que la transformación es conforme en $z_0$, si $\Omega=\omega$.

Es entonces cuando podemos intuir hacia donde nos dirigimos, más allá de la preservación de ángulos en la inversión, es la conservación de otros elementos, de aquí que surge el siguiente teorema.

\begin{teorema}
    Cada inversión lleva circunferencias en circunferencias, circunferencias reales (incluidas las rectas) en circunferencias reales y circunferencias imaginarias en circunferencias imaginarias. 
\end{teorema}

\begin{proof}
    Suponemos que la circunferencia fundamental (la circunferencia bajo la que se hace la inversión) $\mathfrak{C}_0$ de la inversión es una circunferencia real propia (|$\mathfrak{C}_0|<0, A_0 \neq 0$). Para el caso de una línea recta debemos tener en cuenta que $A=0$ y por tanto la ecuación de la recta está dada por $Bz+C\bar{z}+D=0$, sustituimos la definición de inversión obteniendo $Bz^*+c\bar{z^*}+D=0$ y el discriminante de su matriz es $\Delta=-BC=|B|^2$. Es decir, $\Delta^*=\Delta$ esto nos lleva a que transforma rectas en rectas. Debido a la definición geométrica de la inversión, sus propiedades geométricas no dependen de la posición de $\mathfrak{C}_0$ en el plano ni de la magnitud del radio $\rho_0$. Por tanto, podemos elegir $\gamma_0=0$ y $\rho_0=1$. Por \eqref{eq:(16)} la inversión viene dada por $z^*=1/\bar{z}$. Sea $\mathfrak{C}$ cualquier circunferencia. Su imagen por inversión se obtiene sustituyendo $z=1/\bar{z^*}$ en \eqref{eq:(2)}. Después de la multiplicación por el factor positivo $z^*\bar{z^*}$ la ecuación será
    \begin{equation}
        z^*\bar{z^*}\mathfrak{C}\left(\frac{1}{\bar{z^*}},\frac{1}{z^*}\right)=Dz^*\bar{z^*}+Bz^*+C\bar{z^*}+A=0.
        \label{eq:(19)}
    \end{equation}
    Esta es la ecuación de la imagen de la circunferencia $\mathfrak{C}^*=\begin{pmatrix}
        D & B \\
        C & A
    \end{pmatrix}$  y su discriminante cumple que $\Delta^*=\Delta$. Así $\mathfrak{C^*}$ es real sí $\mathfrak{C}$ es real e imaginario sí $\mathfrak{C}$ es imaginario.
\end{proof}
    
\begin{teorema}
    La inversión es una transformación isogonal que cambia el ángulo entre dos curvas en el ángulo negativo
    \begin{equation}
        \omega^*=-\omega
        \label{eq:(20)}
    \end{equation}
\end{teorema}

\begin{proof}
    Para dos circunferencias $\mathfrak{C}_1,\mathfrak{C}_2$ encontramos las imágenes $\mathfrak{C}_1^*,\mathfrak{C}_2^*$ según \eqref{eq:(19)}. Entonces $\Delta_1^*=\Delta_1$, $\Delta_2^*=\Delta_2$, $\Delta_{12}^*=\Delta_{12}$, por lo tanto por \eqref{eq:(14)} tenemos que
    \begin{equation}
        \cos{\omega}^*=\cos{\omega},
        \label{eq:(21)}
    \end{equation}
    sin cambio de signo, pues la inversión convierte la orientación de cualquier circunferencia real en la inversa. Esto puede deducirse fácilmente de la construcción geométrica indicada en la figura \ref{fig:inversion}. Así, $\omega^*=\pm\omega$. sin embargo, con el mismo argumento se concluye que $\omega^*=-\omega$
\end{proof}
\begin{figure}[!ht]
    \begin{center}
        \begin{tikzpicture}
            \clip(-5.931801106481799,-3.876747489925547) rectangle (11.324397240625643,3.4125087084215435);
\draw [line width=2pt] (0,0) circle (4cm);
\draw [line width=2pt] (0,0)-- (7.754149306741345,-1.9015408783553045);
\draw [line width=2pt] (1.4156366869965078,-0.3471549131615304)-- (3.5453639211950914,1.8521324645629875);
\draw [line width=2pt] (1.4156366869965078,-0.3471549131615304)-- (4,0);
\draw [line width=2pt,dash pattern=on 1pt off 1pt] (4.552988317650155,-1.2544495670198967) circle (3.265908578525289cm);
\draw [shift={(4.230291946380404,-4.201959108415395)},line width=2pt]  plot[domain=0.5783498525501367:1.6255474037467348,variable=\t]({1*4.2082650497993574*cos(\t r)+0*4.2082650497993574*sin(\t r)},{0*4.2082650497993574*cos(\t r)+1*4.2082650497993574*sin(\t r)});
\draw [shift={(2.3989801414925207,-3.6696192698558967)},line width=2pt]  plot[domain=0.31889449097912126:1.366092042175719,variable=\t]({1*5.639498026147639*cos(\t r)+0*5.639498026147639*sin(\t r)},{0*5.639498026147639*cos(\t r)+1*5.639498026147639*sin(\t r)});
\begin{scriptsize}
\draw [fill=black] (0,0) circle (2pt);
\draw[color=black] (0.1343145959975415,0.3215996175124599) node {$O$};
\draw [fill=black] (3.5453639211950914,1.8521324645629875) circle (2.5pt);
\draw [fill=black] (4,0) circle (2.5pt);
\draw [fill=black] (7.754149306741345,-1.9015408783553045) circle (2.5pt);
\draw[color=black] (8,-2) node {$z_0^*$};
\draw [fill=black] (1.4156366869965078,-0.3471549131615304) circle (2.5pt);
\draw[color=black] (1,-0.5) node {$z_0$};
\draw[color=black] (2,0) node {$\omega$};
\draw[color=black] (6.7,-0.5) node {$-\omega$};
\draw[color=black] (2.829355918311592,0.7678806092479962) node {$\mathfrak{C}_2$};
\draw[color=black] (2.7954716207909307,-0.44038385356191921) node {$\mathfrak{C}_1$};
\draw[color=black] (6.283074926576055,-0.15773922546274582) node {$\mathfrak{C}_1^*$};
\draw[color=black] (6.316132777815724,0.833996311727335) node {$\mathfrak{C}_2^*$};
\end{scriptsize}
        \end{tikzpicture}
    \end{center}
    \caption{Inversión de ángulos}
    \label{fig:inversion}
    \end{figure}
\subsection{Relación cruzada}
Para concluir con los preeliminares de números complejos vamos a abordar dos conceptos sencillos que nos serán de utilidad al final de este trabajo.
\begin{itemize}
    \item  \textbf{Relación o razón simple}
    Sean $z_1$, $z_2$, $z_3$ tres números complejos no colineales los tres, entonces la relación simple esta definida como
    \begin{equation}
        (z_1;z_2,z_3) = \left\{
        \begin{array}{lcc}
           \frac{z_1-z_2}{z_1-z_3}  & \text{ sí } & z_3 \neq z_1  \\
             \infty & \text{ sí } & z_1=z_3
        \end{array}
        \right.
        \label{eq:(34)}
    \end{equation}
    No abordaremos de manera detallada en esta propiedad de los números complejos pues queda fuera de los parámetros de este trabajo.
    \item \textbf{Relación cruzada}
    Sean $z_1$, $z_2$, $z_3$ tres números complejos no colineales los tres, entonces la relación cruzada esta definida como
    \begin{equation}
        (z_1,z_2;z_3,z_4) = \frac{(z_1-z_3)(z_2-z_4)}{(z_1-z_4)(z_2-z_3)}
        \label{eq:(35)}
    \end{equation}
    De la misma manera no vamos a profundizar, pero estas definiciones nos serán de utilidad en uno de los teoremas fuertes de este trabajo.
\end{itemize}

\newpage

\section{Propiedades de las transformaciones de Möbius}\label{cap:3}
\subsection{Propiedades básicas}\label{cap:3.1}
Las transformaciones de Möbius son funciones que aplican de un $z$-plano (complejo) al $Z$-plano (extendido) $\mathbb{C}^*$. Estas funciones tienen la forma
\begin{equation}
    Z=\mathfrak{H}(z)=\frac{az+b}{cz+d}.
    \label{eq:(22)}
\end{equation}
A esta transformación podemos asociarle una matriz compuesta por los coeficientes constantes de la transformación
\begin{equation*}
    \mathfrak{H}=
    \begin{pmatrix}
        a & b\\
        c & d
    \end{pmatrix}.
\end{equation*}
Cada una de las componentes de la matriz es un número constante complejo y su determinante es 
\begin{equation*}
    |\mathfrak{H}|=\delta=ad-bc
\end{equation*}
 con $\delta$ distinto de cero, pues de serlo tendríamos que $ad-bc=0$, $ad=bc$ y $a = \frac{bc}{d}$, por tanto
\begin{align*}
    Z&=\frac{\frac{bc}{d}z+b}{cz+d}\\
    &=\frac{b\left(\frac{c}{d}z+1\right)}{cz+d}\\
    &=\frac{b(cz+d)}{d(cz+d)}\\
    &=\frac{b}{d}.
\end{align*}
De este modo, la transformación es contante. La transformación $\mathfrak{H}$ evidentemente define la transformación de Möbius, mientras la transformación define la matriz salvo por un factor constante $q\mathfrak{H}$ con $q$ cualquier complejo distinto de cero.

La inversión de la transformación de Möbius es una correspondencia uno a uno entre el $z$-plano y el $Z$-plano, donde este último puede considerarse como una segunda copia del $z$-plano. Sí ambos planos coinciden podemos decir que la transformación representa un mapeo del plano extendido en sí mismo, a diferentes valores de $z$ le corresponden diferentes valores de $Z$ y viceversa.

El punto $z_{\infty}=-d/c$ es llamado polo norte de la función $\mathfrak{H}(z)$. Esta imagen en el $Z$-plano es el punto $Z=\infty$

Una transformación de Möbius es llamada homógrafa en la variable $z$, es decir que preserva las razones de distancias, esto debido a lo siguiente.

En lugar de las variables $z$ y $Z$ introducimos las variables homogéneas $z_1,z_2;Z_1,Z_2$ respectivamente, tales que
\begin{equation*}
    z=\frac{z_1}{z_2} \quad Z=\frac{Z_1}{Z_2}.
\end{equation*}
De la relación \eqref{eq:(22)} podemos obtener
\begin{align}
    \frac{1}{q}Z_1&=az_1+bz_2\notag\\
    \frac{1}{q}Z_2&=cz_1+dz_2
    \label{eq:(23)}
\end{align}
Lo anterior podemos verlo de la siguiente manera
\begin{align*}
    Z&=\frac{az+b}{cz+d}\\
     &=\frac{a\left(\frac{z_1}{z_2}\right)+b}{c\left(\frac{z_1}{z_2}\right)+d}\\
     &=\frac{\frac{az_1+bz_2}{z_2}}{\frac{cz_1+dz_2}{z_2}}\\
     \frac{Z_1}{Z_2}&=\frac{az_1+bz_2}{cz_1+dz_2}.
\end{align*}
Dado que sí $\frac{a}{b}=\frac{c}{d}$ de donde $a=\alpha c$ y $b=\alpha d$ con $\alpha \in \mathbb{Q}$ entonces
\begin{equation*}
    \frac{Z_1}{q}=az_1+bz_2 \text{ y } \frac{Z_2}{q}=cz_1+dz_2,
\end{equation*}
donde $q$ es un número complejo arbitrario diferente de cero. Esto es una transformación homogénea en variables $z_1$ y $z_2$, luego hacemos uso de la matriz $\mathfrak{H}$ y la siguiente notación columna
\begin{equation*}
    \mathfrak{z}=\frac{z_1}{z_2}, \quad \mathfrak{Z}=\frac{Z_1}{Z_2}.
\end{equation*}
De manera que \eqref{eq:(23)} se puede ver de la siguiente manera
\begin{equation}
    \mathfrak{Z}=q\mathfrak{H}\mathfrak{z}.
    \label{eq:(24)}
\end{equation}
Finalmente es llamada "transformación bilineal" porque esta deriva de la relación bilineal entre dos variables $z$ y $Z$
\begin{equation}
    czZ+dZ-az-b=0.
    \label{eq:(25)}
\end{equation}
\textbf{Nota: }Se puede denominar "transformación bilineal" debido a la relación bilineal que subyace en su definición. La ecuación \eqref{eq:(25)} es una representación implícita de la transformación de Möbius, donde $z$ y $Z$ son dos variables complejas. En esta ecuación, la relación entre $z$ y $Z$ es de naturaleza bilineal, ya que ambas variables aparecen en términos lineales y en productos cruzados.

Sin embargo, es importante destacar que, aunque esta representación implícita muestra una relación bilineal entre las variables $z$ y $Z$, la expresión directa de una transformación de Möbius \eqref{eq:(22)}, es no lineal en términos de $z$ debido a la división de números complejos. Por lo tanto, aunque se pueda hablar de una relación bilineal subyacente en las transformaciones de Möbius, no son estrictamente bilineales en el sentido en que se usan comúnmente en álgebra lineal. La terminología puede variar según el contexto, pero en general, se les llama transformaciones de Möbius o transformaciones homográficas.

Para recalcar, la ecuación \eqref{eq:(16)} que define una inversión
\begin{equation*}
    z^*=\frac{C_0\bar{z}+D_0}{A_0\bar{z}+B_0}
\end{equation*}
no se categoriza como una transformación de Möbius dado que no sería una transformación bilineal ni proyectiva. Este es un caso especial de una transformación de la forma $Z=\mathfrak{H}(\bar{z})$ la cual es llamada anti-homógrafa. Para entender este concepto vamos a entrar en dos aspectos importantes, las transformaciones de Möbius como un grupo y el segundo las trasnformaciones de Möbius de tipo simple.
\subsection{El grupo de las transformaciones de Möbius}\label{cap:3.2}
Sean $\mathfrak{H}_1,\mathfrak{H}_2$ dos transformaciones de Möbius
\begin{equation*}
    \mathfrak{H}_1=
    \begin{pmatrix}
        a_1 & b_1\\
        c_1 & d_1
    \end{pmatrix}
    \text { y } 
    \quad
    \mathfrak{H}_2=
    \begin{pmatrix}
        a_2 & b_2\\
        c_2 & d_2
    \end{pmatrix}.
\end{equation*}
Sí estas transformaciones son llevadas a cabo en sucesión, primero $z_1=\mathfrak{H}_1(z)$, entonces $Z=\mathfrak{H}_2(z_1)$ y obtenemos la transformación
\begin{equation}
    Z=\mathfrak{H}_2(\mathfrak{H}_1(z)).
    \label{eq:(26)}
\end{equation}
Esta transformación es entendida como el producto de dos transformaciones $\mathfrak{H}_1,\mathfrak{H}_2$ en ese orden. Esta es a su vez una transformacion de Möbius $\mathfrak{H}_3$
\begin{equation*}
    Z=\frac{a_2\mathfrak{H}_1(z)+b_2}{c_2\mathfrak{H}_1(z)+d_2}=\frac{(a_1a_2+b_2c_1)z+a_2b_1+b_2d_1}{(c_2a_1+d_2c_1)z+c_2b_1+d_2d_1}.
\end{equation*}
Notemos que la matriz $\mathfrak{H}_3$ se puede ver como el producto de las matrices $\mathfrak{H}_1$ y $\mathfrak{H}_2$
\begin{equation}
    \mathfrak{H}_3=\mathfrak{H}_2\mathfrak{H}_1=
    \begin{pmatrix}
        a_2a_1+b_2c_1 & a_2b_1+b_2d_1\\
        c_2a_1+d_2c_1 & c_2b_1+d_2d_1
    \end{pmatrix}.
    \label{eq:(27)}
\end{equation}
La función producto \eqref{eq:(26)} es con certeza no constante y su inversa a su vez es una transformación de Möbius
\begin{equation}
    z=(H)^{-1}(Z)=\frac{dZ-b}{-cZ+a}
    \label{eq:(28)}
\end{equation}
y su matriz inversa omite el factor $1/\delta$
\begin{equation*}
    \begin{pmatrix}
        d & -b\\
        -c & a
    \end{pmatrix}.
\end{equation*}
Teniendo en cuenta que $\mathfrak{H}=\begin{pmatrix}
    a & b\\
    c & d
\end{pmatrix}$, tenemos que $\mathfrak{H}\mathfrak{H}^{-1}=\mathfrak{I}_z$. En efecto,
\begin{align*}
    \begin{pmatrix}
        a & b\\
        c & d
    \end{pmatrix}
    \begin{pmatrix}
        d & -b\\
        -c & a
    \end{pmatrix} 
    &=
    \begin{pmatrix}
        ad-bd & -ab+ab \\
        cd-dc & -cb+da
    \end{pmatrix}\\
    &=
    \begin{pmatrix}
        ad-bc & 0 \\
        0 & ad-bc
    \end{pmatrix}\\
    &=(ad-bc)
    \begin{pmatrix}
        1 & 0 \\
        0 & 1
    \end{pmatrix}.
\end{align*}
Se hace evidente que el factor no afecta a la identidad, pues sabemos que las transformaciones son equivalentes salvo un factor escalar.

Una transformación lineal de Möbius especial es la identidad $Z=z$ que viene dad por la matriz identidad
\begin{equation*}
    \mathfrak{I}=
    \begin{pmatrix}
        1 & 0 \\
        0 & 1
    \end{pmatrix}
\end{equation*}
El producto de matrices es asociativa, la misma es por lo tanto válida para tres o más transformaciones de Möbius.
\begin{teorema}
    El sistema de todas las transformaciones de Möbius es un grupo con la composición funcional como un grupo multiplicativo.
\end{teorema}
Por medio de la proyección estereográfica, cada transformación de Möbius puede ser transferida a la esfera. Aparece allí como cierta transformación de la esfera sobre sí misma. El grupo de todas las transformaciones esféricas es isomorfo al grupo de las transformaciones de Möbius.

La representación analítica de las transformaciones esféricas se sale del objetivo de este trabajo.

\subsection{Transformaciones de Möbius de tipo simple}\label{cap:3.3}
Sí $c=0$ podemos asumir que $d=1$, con lo cual $\mathfrak{H}(z)$ se puede ver como una función lineal
\begin{equation*}
    Z=az+b.
\end{equation*}
Sí $c\neq0$ la fórmula \eqref{eq:(22)} puede ser escrita de la siguiente forma haciendo la división de polinomios
\begin{equation*}
\begin{array}{lrcl}
     &az+b &=& \frac{a}{c}\left(cz+d\right)+\left(b-\frac{ad}{c}\right).\\
    \text{De donde} & \frac{az+b}{cz+d} &=& \frac{a}{c}+\frac{bc-ad}{c(cz+d)},
\end{array}
\end{equation*}
\begin{equation}
    \text{con lo que} \quad Z = \frac{a}{c}-\frac{\delta}{c}\frac{1}{cz+d}. 
    \label{eq:(29)}
\end{equation}
De este modo una transformación de Möbius aparece como un producto de transformaciones de Möbius de tipo simple
\begin{enumerate}
    \item Translación
    \begin{equation}
        Z=z+b \quad \text{ con matriz asociada } \mathfrak{I}_b=\begin{pmatrix}
            1 & b\\
            0 & 1
        \end{pmatrix} \text{ con } b\neq 0 \text{ complejo }
        \label{eq:(30)}
    \end{equation}
    Cada recta paralela al vector translación $\overrightarrow{0b}$ es invariante o transformado en sí mismo por la traslación \eqref{eq:(30)}. Las líneas de cada haz son intercambiadas en particular, esto es válido para un haz perpendicular al vector $\overrightarrow{0b}$.
    \begin{figure}[!ht]
        \begin{center}
            \begin{tikzpicture}
                \clip(-4,-4.5) rectangle (9,6.3);
                    \draw [-latex',line width=2pt] (0,0) -- (1,3);
                    \draw [line width=2pt,domain=-6.121042670321929:11.004975779907687] plot(\x,{(-0--1*\x)/-3});
                    \draw [line width=2pt,domain=-6.121042670321929:11.004975779907687] plot(\x,{(--11-3*\x)/-1});
                    \draw [line width=2pt] (3.3,-1.1) circle (2.213594362117866cm);
                    \draw [line width=2pt,domain=-6.121042670321929:11.004975779907687] plot(\x,{(-5--1*\x)/-3});
                    \draw [line width=2pt] (4,1) circle (4.427188724235733cm);
                    \begin{scriptsize}
                    \draw [fill=black] (0,0) circle (2pt);
                    \draw [color=black] (1.1,3) node {$b$};
                    \draw[color=black] (0.6,1.1) node {$\frac{b}{2}$};
                    \draw [fill=black] (0.5,1.5) circle (2pt);
                    \draw [fill=black] (4,1) circle (2.5pt);
                    \draw[color=black] (4.2,1.1) node {$z$};
                    \draw[color=black] (-3,1.7) node {$\bar{b}z+b\bar{z}=0$};
                    \draw[color=black] (-3,3.4) node {$\bar{b}z+b\bar{z}=b\bar{b}$};
                    \draw [fill=black] (2.6,-3.2) circle (2pt);
                    \draw [fill=black] (2.6,-3.2) circle (2pt);
                    \draw[color=black] (3,-4) node {$z^*=-\frac{b}{\bar{b}}\bar{z}$};
                    \draw [fill=black] (5.4,5.2) circle (2pt);
                    \draw[color=black] (6.1,5.547002568071058) node {$Z=z+b$};
                    \end{scriptsize}
                \end{tikzpicture}
        \end{center}
        \caption{Translación del punto $z$ al punto $Z=z+b$}
        \label{fig:translacion}
        \end{figure}
    Cada uno de estos haces corresponde en la esfera a un haz parabólico de circunferencias, todos pasando por el punto $S$. Este es el único punto invariante por la transformación  esférica que corresponde a \eqref{eq:(30)}. Por esto $z=\infty$ es el único punto construido de la translación. La translación $\mathfrak{I}_b$ es el producto (que resulta de la composición de dos inversiones) (Figura \ref{fig:translacion})
    \begin{proof}
        El aspecto geométrico sugiere elegir lineas rectas como circunferencias fundamentales de la inversión, estas líneas deben ser perpendiculares al vector translación $\overrightarrow{0b}$. Para la primera inversión podemos tomar la línea a través de $0$ dada por la ecuación $\bar{b}z+b\bar{z}=0$, como inversión obtenemos
        \begin{equation*}
            z^*=-\frac{b}{\bar{b}}\bar{z}.
        \end{equation*}
        Como segundo cículo fundamental elegimos la línea $\bar{b}z+b\bar{z}=b\bar{b}$ que tiene una distancia $|b|/2$ de $0$ medida en dirección de $\overrightarrow{0b}$ la correspondiente inversión de $z^*$ es 
        \begin{align*}
            Z &= -\frac{b\bar{z^*}-b\bar{b}}{\bar{b}}\\
            &= \frac{\bar{b}z+\bar{b}b}{\bar{b}}\\
            &= z+b.
        \end{align*}
    \end{proof}
    \item Rotación
    \begin{equation}
        Z=e^{i\alpha}z \quad \text{ con matriz asociada }\mathfrak{R}_\alpha =
        \begin{pmatrix}
            e^{i\alpha} & 0\\
            0 & 1
        \end{pmatrix}
        \label{eq:(31)}
    \end{equation}
    Cada circunferencia con centro en $0$ es invariante. Son intercambiados las circunferencias del haz que son ortogonales al haz de circunferencias concéntricas que es el haz de todas las líneas que pasan por $0$. Las rotaciones sobre $0$ en el plano corresponden a las rotaciones de la esfera sobre el eje polar $NS$.

    Sí $\alpha$ no es un múltiplo entero de $2\pi$ entonces $N$ y $S$ son los únicos puntos construidos de la esfera de rotaciones, en consecuencia, $0$ y $\infty$ son los únicos puntos fijos de la rotación \eqref{eq:(31)}.

    Cada rotación es el producto de dos inversiones
    \begin{figure}[ht]
        \begin{center}
            \begin{tikzpicture}
                \clip(-6,-6) rectangle (6,6);
                \draw [shift={(0,0)},line width=2pt,color=green,fill=green,fill opacity=0.10000000149011612] (0,0) -- (0:0.5454545454545452) arc (0:53.13010235415597:0.5454545454545452) -- cycle;
                \draw [shift={(0,0)},line width=2pt,color=red,fill=red,fill opacity=0.1] (0,0) -- (21.80140948635181:0.5454545454545452) arc (21.80140948635181:128.06161419466378:0.5454545454545452) -- cycle;
                \draw [line width=2pt] (0,0) circle (5.385164807134505cm);
                \draw [line width=2pt,dotted] (4,0) circle (2.23606797749979cm);
                \draw [line width=2pt] (0,0)-- (5,2);
                \draw [line width=2pt] (0,0)-- (5,-2);
                \draw [line width=2pt] (0,0)-- (3.231098884280703,4.308131845707603);
                \draw [line width=2pt,dash pattern=on 1pt off 1pt] (3.231098884280703,4.308131845707603) circle (6.5514531624688725cm);
                \draw [line width=2pt] (-3.32,4.24)-- (0,0);
                \draw [line width=2pt,domain=-8.798181818181813:8.98363636363636] plot(\x,{(-0-0*\x)/-4});
                \begin{scriptsize}
                \draw [fill=black] (0,0) circle (2pt);
                \draw[color=black] (-0.1618181818181803,-0.3) node {$0$};
                \draw [fill=black] (5,2) circle (2.5pt);
                \draw[color=black] (5.1472727272727266,2.3945454545454523) node {$z$};
                \draw [fill=black] (5,-2) circle (2.5pt);
                \draw[color=black] (5.4,-2.2963636363636346) node {$z^*=\bar{z}$};
                \draw [fill=black] (3.231098884280703,4.308131845707603) circle (2.5pt);
                \draw[color=black] (3.3836363636363638,4.70363636363636) node {$re^{i\frac{1}{2}\alpha}$};
                \draw[color=black] (2.1,0.3) node {$\frac{1}{2}\alpha = 53.13$};
                \draw [fill=black] (-3.32,4.24) circle (2pt);
                \draw[color=black] (-2.5,4.3) node {$Z=ze^{i\alpha}$};
                \draw[color=black] (0,0.94) node {$\alpha = 106.26$};
                \end{scriptsize}
            \end{tikzpicture}
        \end{center}
        \caption{Rotación del punto $z$ al punto $Z=ze^{i\alpha}$}
        \label{fig:rotacion}
        \end{figure}
        \begin{proof}
            Como circunferencia fundamental para la primera inversión elegimos el eje real $iz-i\bar{z}=0$. Por esto $z^*=\bar{z}$. Como segunda circunferencia tomamos la línea a través de $0$ la cual hace el ángulo $\frac{1}{2}\alpha$ con el eje real
            \begin{equation*}
                ie^{-i(\alpha/2)}z-ie^{i(\alpha/2)}\bar{z}=0.
            \end{equation*}
            La segunda inversión toma $z^*$ en 
            \begin{align*}
                Z&=(e^{i(\alpha/2)}/e^{-i(\alpha/2)})\bar{z^*}\\
                &=\frac{e^{i(\alpha/2)}}{e^{-i(\alpha/2)}}\bar{z^*}\\
                &=e^{i(\alpha/2)}e^{i(\alpha/2)}\bar{\bar{z}}\\
                &=e^{i\alpha}z.
            \end{align*}
        \end{proof}
        \item Dilatación alrededor de $0$ con coeficiente de extensión $\rho>0$
        \begin{equation}
            Z=\rho z \quad \text{ con matriz asociada }\mathfrak{D}_\rho =
            \begin{pmatrix}
                \rho & 0\\
                0 & 1
            \end{pmatrix}
            \label{eq:(32)}
        \end{equation}
        Cada recta que pasa por el origen es invariante. Todos las circunferencias alrededor de $0$ están intercambiadas. Nuevamente los puntos fijos son $0$ y $\infty$. Toda dilatación es producto de dos inversiones.
        \begin{proof}
            Como primer circunferencia fundamental tomemos la circunferencia unitaria alrededor de $0$
            \begin{equation*}
                z^*=\frac{1}{\bar{z}}.
            \end{equation*}
            Para la segunda inversión tomemos la circunferencia con el radio $\sqrt{\rho}$ sobre $0$. Entonces
            \begin{equation*}
                Z=\frac{\rho}{\bar{z^*}}=\rho z.
            \end{equation*}
        \end{proof}
        \item Reciprocidad
        \begin{equation}
            Z=\frac{1}{z} \text{con matriz asociada } \mathfrak{R}=
            \begin{pmatrix}
                0 & 1\\
                1 & 0
            \end{pmatrix}
            \label{eq:(33)}
        \end{equation}
        En la esfera, la reciprocidad lleva el punto $P(\xi,\eta,\zeta)$ en el punto $(\xi,-\eta,\zeta)$; pues sí $z=(\xi+i\eta)/(1+\zeta)$, entonces $1/z=(\xi-i\eta)/(1-\zeta)$ Por lo tanto, en la esfera la reciprocidad es una rotación de 180° alrededor del eje $\xi$.

        Todas las circunferencias en planos paralelos a $\eta$ y todas las circunferencias máximos cuyos planos contenidos el eje $\xi$ son invariantes. Así, cada circunferencia a través de $1$ y $-1$ y cada circunferencias del haz hiperbólico ortogonal se transforma sobre sí mismo.

        De la interpretación de la reciprocidad como una rotación de la esfera es obvio que también la reciprocidad es producto de dos inversiones.
\end{enumerate}
\subsection{Propiedades de la transformaciones de Möbius}\label{cap:3.4}
La notación matricial de los tipos simples de transformaciones de Möbius nos permite escribir la matriz de una transformación lineal de Möbius en la forma
\begin{equation*}
    \mathfrak{H}=\mathfrak{I}_b\mathfrak{R}_\alpha\mathfrak{D}_{|a|},
\end{equation*}
y en virtud de \eqref{eq:(29)} tenemos que si $c=|c|e^{i\gamma}\neq0$ y $-\delta/c=|\delta/c|e^{i\phi}$. Si hacemos el producto matricial obtenemos 
\begin{align*}
    \mathfrak{H}&=
    \begin{pmatrix}
        1 & b\\
        0 & 1
    \end{pmatrix}
    \begin{pmatrix}
        e^{i\alpha} & 0\\
        0 & 1
    \end{pmatrix}
    \begin{pmatrix}
        |a| & 0\\
        0 & 1
    \end{pmatrix}
    \\
    &=
    \begin{pmatrix}
        e^{i\alpha} & b\\
        0 & 1
    \end{pmatrix}
    \begin{pmatrix}
        |a| & 0\\
        0 & 1
    \end{pmatrix}
    \\
    &=
    \begin{pmatrix}
        |a|e^{i\alpha} & b\\
        0 & 1
    \end{pmatrix}.
\end{align*}
De modo que la transformación de Möbius $\mathfrak{H}(z)=|a|e^{i\alpha}z+b$ es lineal.

En el caso no lineal podemos ver la transformación de la siguiente forma
\begin{equation*}
    \mathfrak{H}=\mathfrak{I}_{a/c}\mathfrak{D}_{|\delta/c|}\mathfrak{R}\mathfrak{I}_{d}\mathfrak{D}_{|c|}\mathfrak{R}_{\gamma}
\end{equation*}
Así, cada transformación de Möbius aparece como producto de no más que siete transformaciones de los cuatro tipos simples en un cierto orden. Por lo tanto, es producto de un número par (menor o igual a catorce) de inversiones.

De los teoremas de la sección \ref{cap:2} concluimos
\begin{teorema}
    Cada transformación de Möbius es una aplicación conforme del plano $z$ completo en el plano $Z$ completo. Lleva circunferencias en circunferencias, circunferencias reales en circunferencias reales (incluyendo líneas rectas) y circunferencias imaginarias en circunferencias imaginarias
\end{teorema}
Según la sección \ref{cap:2}, a toda circunferencia $\mathfrak{C}$ se le puede dar una orientación. Es geométricamente evidente que el mapeo con una transformación de Möbius de uno de los tres primeros tipos (translación, rotación y dilatación) conserva la orientación de cada circunferencia en el plano. Su interior se mapea en el interior de la imagen de la circunferencia.

La construcción geométrica del punto $1/z$ para un $z\neq 0$ dado, muestra que la reciprocidad conserva la orientación de una circunferencia que no contiene $0$, pero invierte la orientación de cualquier circunferencia que contiene $0$ como un punto interior. Por esta razón tenemos el siguiente teorema.

\begin{teorema}
    Toda transformación de Möbius $\mathfrak{H}$ conserva la orientación de cualquier circunferencia que no contenga el polo de $\mathfrak{H}$. Sí $\mathfrak{C}$ contiene el polo de $\mathfrak{H}$ entonces la imagen $\mathfrak{C}_1$ de $\mathfrak{C}$ tiene su orientación opuesta a la de $\mathfrak{C}$. Sí el polo de $\mathfrak{H}$ se encuentra en $\mathfrak{C}$, entonces la imagen de $\mathfrak{C}$ es una línea recta.
\end{teorema}
Cualquier anti-homografía se obtiene siguiendo la inversión $z^*=\bar{z}$ con respecto al eje real por una transformación de Möbius adecuada, por tanto una anti-homografía se obtiene como producto de un número impar (menor a quince) de inversiones. Se trata por tanto de una transformación isogonal que invierte el sentido de giro. Esta aplica circunferencias a circunferencias.
\begin{teorema}
    Sean $z_1,z_2,z_3,z_4$ cuatro puntos cualesquiera del plano completo de los cuales tres no colineales. Sean $\mathfrak{H}$ una transformación de Möbius y $Z_j=\mathfrak{H}(z_j)$ con $j=1,2,3,4$ entonces
    \begin{equation*}
        (Z_1,Z_2;Z_3,Z_4)=(z_1,z_2;z_3,z_4)
    \end{equation*}
    Más brevemente, la relación cruzada es un invariante (invariante de cuatro puntos) del grupo de todas las transformaciones de Möbius.
\end{teorema}
\begin{proof}
    Sí los cuatro puntos son transformados en $z_j^*$ por una inversión, luego su relación cruzada es dada por su valor conjugado, por tanto, la relación cruzada es invariante bajo un producto de dos o cualquier número par de inversiones.
\end{proof}
\begin{example}
    Supongamos que tenemos cuatro puntos en el plano complejo:
    \begin{equation*}
        z_1,\quad
        z_2, \quad
        z_3, \quad
        z_4.
    \end{equation*}
    
    Aplicamos inversión a cada uno de los puntos $z_1, z_2, z_3, z_4$ para obtener
    \begin{equation*}
        Z_1=z_1^*, \quad Z_2=z_2^*, \quad Z_3=z_3^*, \quad Z_4=z_4^*,    
    \end{equation*}
    por definición tenemos que 
    \begin{align*}
        z^*_j=\gamma_0\frac{\rho_0^2}{\bar{z_j}-\bar{\gamma_0}},
    \end{align*}
    esto quiere decir
    \begin{align*}
        z^*_j-z^*_k=\rho_0^2\frac{\bar{z_k}-\bar{z_j}}{(\bar{z_j}-\bar{\gamma_0})(\bar{z_k}-\bar{\gamma_0})}.
    \end{align*}
    Partiendo de la definición \eqref{eq:(35)} podemos ver 
    \begin{align*}
        (z_1^*,z_2^*;z_3^*,z_4^*)
        &=\frac{\left(\rho_0^2\frac{\bar{z_3}-\bar{z_1}}{(\bar{z_1}-\bar{\gamma_0})(\bar{z_3}-\bar{\gamma_0})}\right)\left(\rho_0^2\frac{\bar{z_4}-\bar{z_2}}{(\bar{z_2}-\bar{\gamma_0})(\bar{z_4}-\bar{\gamma_0})}\right)}{\left(\rho_0^2\frac{\bar{z_4}-\bar{z_1}}{(\bar{z_1}-\bar{\gamma_0})(\bar{z_4}-\bar{\gamma_0})}\right)\left(\rho_0^2\frac{\bar{z_3}-\bar{z_2}}{(\bar{z_2}-\bar{\gamma_0})(\bar{z_3}-\bar{\gamma_0})}\right)}\\
        &=\frac{\frac{(\bar{z_3}-\bar{z_1})(\bar{z_4}-\bar{z_2})}{(\bar{z_1}-\bar{\gamma_0})(\bar{z_3}-\bar{\gamma_0})(\bar{z_2}-\bar{\gamma_0})(\bar{z_4}-\bar{\gamma_0})}}{\frac{(\bar{z_4}-\bar{z_1})(\bar{z_3}-\bar{z_2})}{(\bar{z_1}-\bar{\gamma_0})(\bar{z_4}-\bar{\gamma_0})(\bar{z_2}-\bar{\gamma_0})(\bar{z_3}-\bar{\gamma_0})}}\\
        &=\frac{(\bar{z_3}-\bar{z_1})(\bar{z_4}-\bar{z_2})}{(\bar{z_4}-\bar{z_1})(\bar{z_3}-\bar{z_2})}\\
        &=\frac{\overline{(z_1-z_3)}\overline{(z_2-z_4)}}{\overline{(z_1-z_4)}\overline{(z_3-z_2)}}\\
        &=\overline{\frac{(z_1-z_3)(z_2-z_4)}{(z_1-z_4)(z_3-z_2)}}\\
        &=\overline{(z_1,z_2;z_3,z_4)}.
    \end{align*}
    Al hacer nuevamente la inversión repetimos el mismo proceso obteniendo un doble conjugado, con lo cual se puede ver que es invariante.

    Es necesario comentar que con todo lo anterior podemos crear una transformación de Möbius que cumpla ciertas condiciones requeridas, tomemos como ejemplo una transformación que aplique la circunferencia en sí misma y un punto interno en un punto externo, dado que al hacer una inversión, los puntos externos se transforman en internos y viceversa, así, al aplicar una doble inversión los puntos internos vuelven a ser internos preservando la circunferencia inicial. 
    
\end{example}
\newpage
\section{Conclusiones}
La demostración de los tres teoremas anteriores nos dejan 3 conclusiones claras
\begin{enumerate}
    \item Las transformaciones de Möbius conforman un grupo multiplicativo con la composición como operación binaria.
    \item Las transformaciones de Möbius transforman circunferencias a circunferencias, bien sean líneas, circunferencias reales o imaginarias.
    \item Toda transformación de Möbius conserva la orientación de cualquier circunferencia que no contenga el polo de la circunferencia transformada.
    \item La relación cruzada es un invariante en el grupo de las transformaciones de Möbius.
\end{enumerate}
\newpage

\bibliographystyle{plain}
\bibliography{basededatosbibliografia}

%\printindex

\end{document}